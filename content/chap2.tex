\chapter{使用说明}
\section{格式说明}
本模板的封面样式及内容格式,均遵循青岛大学教务处2015年印发的《青岛大学本科毕业论文(设计)封面》和《青岛大学本科毕业论文(设计)基本规范要求》,详见\url{http://jw.qdu.edu.cn/homepage/infoSingleArticle.do?articleId=4755&columnId=453}。

本模板不包含任务书和评分表,可从上面的链接下载,自行填写后打印。

\section{文件组织结构}
\begin{description}
\item[main.tex] 主文档;
\item[mainref.bib] BibTeX格式的参考文献数据;
\item[bstutf8.bst] 参考文献样式,不需要修改;
\item[content] 存放摘要、各章节及谢辞等文档的目录;
\item[figures] 存放论文中插入的图片的目录。
\end{description}
\section{使用前准备}
在使用本模板编译\LaTeX 文档之前,需安装以下软件:
\begin{itemize}
\item{\bf 方正小标宋字体} 封面标题字体为方正小标宋。可以在\url{http://www.foundertype.com/}购买;
\item{\bf texlive 2016} 本模板基于texlive 2016包含的CTeX宏包编写,无法保证能在更早期的版本上编译。可以从\url{http://tug.org/texlive/}下载安装。
\end{itemize}

还需要掌握\LaTeX 基础知识。可阅读《一份不太简短的\LaTeX 介绍》(\url{http://www.latexstudio.net/archives/6058}),或购买刘海洋编著的《\LaTeX 入门》。

\section{编译说明}
需要对主文档执行四次编译,通过 xelatex + bibtex + xelatex + xelatex 生成带有完整目录和参考文献信息的 PDF 文件。

\section{查重须知}
知网查重仅需要正文和参考文献,可注释掉无关的包含文件代码后编译。必要时可使用pandoc(\url{http://www.pandoc.org/})将\LaTeX 文档 转换为word文档以供查重之用。

\section{后续更新}
由于水平有限,时间紧迫,精力不足,目前本模板仅提供最基本的排版设置,未能提供一份详尽的参考,还可能存在未知的bug。作者仍将继续维护本模板,也希望能有校友参与。

\section{关于作者}
2013-2017年就读于青岛大学计算机科学技术学院(原信息工程学院)计算机科学与技术专业。